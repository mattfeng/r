\documentclass{article}

\usepackage{amsmath}

\newcommand{\pH}{\text{pH}}
\newcommand{\pKa}{\text{p}{K_a}}
\newcommand{\conc}[1]{\left[#1\right]}

\title{Protonation states}
\author{Matthew Feng}

\begin{document}

\maketitle

\section{Introduction}

The Henderson-Hasselbalch equation states that

$$
\pH = \pKa + \log_{10} \frac{\conc{\text{base}}}{\conc{\text{acid}}}.
$$

By rearranging terms, we can write the Henderson-Hasselbalch equation as

$$
\frac{\conc{\text{base}}}{\conc{\text{acid}}} = 10^{\pH - \pKa}.
$$

In this form, we can see that when the $\pH$ of the surroundings is greater than $\pKa$, more base is present. In other words, the {\bf deprotonated form dominates when the acid is more acidic than its surroundings}. Furthermore, for every increase in the difference $\pH - \pKa$, the dominance of the deprotonated form increases ten-fold.

\end{document}
